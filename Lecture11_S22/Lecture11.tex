\documentclass[12pt]{beamer}
\usetheme{Boadilla}
\usepackage{graphicx}
\usepackage{algorithm2e}
\graphicspath{{images/}}
\title{CMPT 155: Computer Applications for Life Sciences}
\subtitle{Lecture 11: Continuous Probability - The Normal Distribtuion}
\author{Ivan E. Perez}
\institute{}
\date{April 19, 2022}
\usepackage{booktabs} % Allows the use of \toprule, 
\usepackage{appendix}
\usepackage{enumerate,multicol}
\usepackage{amsmath, amssymb, amsthm}
\usepackage{tikz}
\usepackage{amsxtra}
\usepackage{bold-extra}
\begin{document}
	
	\begin{frame}
		\titlepage
	\end{frame}
	
	\begin{frame}
		\frametitle{Presentation Outline}
		\tableofcontents
	\end{frame}
	\section{Homework \& Administrative}
	
	\begin{frame}
		\frametitle{Homework \& Administrative Schedule}
		\begin{itemize}
			\item Homeworks:
			\begin{itemize}	
				\item \#6 Due: Friday April 22\textsuperscript{th} at 6pm
				\item \#7 Due: Friday, April 29 \textsuperscript{rd} at 6pm
				\item \#8 Due: Friday, May 6\textsuperscript{th} at 6pm
			\end{itemize}
			\item Final Exam Review: Tuesday, May 3\textsuperscript{rd} at 6pm
			\item Mock Final Exam: Wednesday, May 4\textsuperscript{th}
			\item \textbf{Final Exams}:
			\begin{itemize}
				\item Section 01 (8am) Final Exam: May 9\textsuperscript{th} 11am - 1pm
				\item Section 02 (9am) Final Exam: May 10\textsuperscript{th} 11 am - 1pm
			\end{itemize}
		\end{itemize}
	\end{frame}
\section{Introduction to Continuous Random Variables and the Normal Distribution}
	\begin{frame}
		\frametitle{Differences Between Discrete and Continuous Random Variables}
		\begin{itemize}
			\item Discrete probability is computed by \textit{counting} the subset of outcomes that satisfy restrictions against the broader set of outcomes.
			\item Coninuous probability is about \textit{measuring} the range of outcomes that satisfy restrictions against the broader range of outcomes.
		\end{itemize}
	Example: 
	\begin{itemize}
		\item In the Binomial Model for Probability we were interested in the \textbf{direcete} number of $k$-success in $n$-trials.
		\item In the Poisson Model for Probability we were interested in the \textbf{discrete} number of $x$-arrivals in a period of length $t$.
		\item In the Guassian(Normal) Model for Probability we are interested in \textbf{ranges} of values.  
	\end{itemize}	
	\end{frame}
	
	\begin{frame}
		\frametitle{Why Should I care about the Normal Distribution?}
		While few random variables are normally distributed on their own. The Central Limit Theorem states that the sum of non-normally distributed random variables tends to turn into a normal distribution as more random variables are added to the sum. \\
		
		Example: Adding Die to a dice roll
		\begin{center}
		\includegraphics[width=0.5\textwidth]{DicePlots_770.png}
		\end{center}
		
		
	\end{frame}

	\begin{frame}
		\frametitle{Examples/Descriptions of the Normal Distribution}
		Common Examples of Normal Distributions include:
		\begin{itemize}
			\item Distribution of Heights at different ages
			\item Distribution of food weights at a deli.  
			\item Distribution of Test scores. 
		\end{itemize}
	\bigskip
	The Normal Distribution has the following characterstics:
	\begin{itemize}
		\item Symmetric about the Mean.
		\item The Mode, Mean, and Median are all equivalent values.
		\item 50 \% of values lie below the Mean, and 50 \% of values lie above the mean.
		\item long tails that stretch infinitely. 
	\end{itemize}
	\end{frame}
\section{Defining the Normal Distribution}
	\subsection{Parameters of a Normal Distribution}
	\begin{frame}
		\frametitle{Features/Parameters of the Normal Distribution}
		
	Normally distributed random variables can be described using the parameters for location, $\mu$, and dispersion $\sigma^{2}$.
	\begin{itemize}
		\item Mean - $\mu$
		\begin{itemize}
			\item Measure of central tendency/location
			\item Can be estimated using AVERAGE(), 
		\end{itemize}
		\item Variance - $\sigma^2$
		\begin{itemize}
			\item Measure of dispersion/scale
			\item Can be estimated using VAR() or VAR.S()
			\item its cousin Standard Deviation, $\sigma$
			\begin{itemize}
				\item Can be estimated using STDEV()
				\item is just $\sqrt{\sigma^2}$
			\end{itemize}
		\end{itemize}
	\end{itemize}
	\end{frame}
\subsection{Equations of the Normal Distribution}
	\begin{frame}
		\frametitle{Equations of the Normal Distribution}
		Probability density functions(pdf's) can be used to describe distributions.
		For a given $\mu, \sigma$, the normal distribution's probability density function is \begin{equation} 
			f(x,\mu,\sigma)= \frac{1}{\sigma\sqrt{2\pi}}e^{-\frac{1}{2}\left(\frac{x-\mu}{\sigma}\right)^2}. 
			\end{equation}
		Where $x$ is the input variable whose probability density is being computed using $f(x,\mu,\sigma)$. \\
		The probability of a range of values is the area under the curve(AUC) between two computed probability densities.\\ The probability of a random variable, $\textbf{X}$, being within the range of $x_{1}$, and $x_{2}$, (i.e., $[x_1,x_2]$), for a given $\mu, \sigma$ is:
		\begin{equation}
		 P(x_{1} \leq \textbf{X} \leq x_{2}) 
		= \int_{x_{1}}^{x_{2}} f(\textbf{X}, \mu, \sigma)dx 
		\end{equation}
	\end{frame}
	\begin{frame}
	\frametitle{Equations of the Normal Distribution}
	Unfortunately, NORMDIST() only computes probabilities from $-\infty$ to $x$ in the form:
	\begin{equation}
		P(\textbf{X}\leq x) =  \int_{-\infty}^{x} f(x, \mu, \sigma)dx.
	\end{equation}
	We can express (Eq. 2) as a difference of probabilities,
	\begin{equation}
		P(x_1 \leq \textbf{X} \leq x_2 ) = P(\textbf{X}\leq x_2) - P(\textbf{X} \leq x_1)
	\end{equation}
	To express probabilities where the random variable, $\textbf{X}$, is at least $x_1$; we take the complement of $P(\textbf{X}\leq x_1)$:
	\begin{equation}
		P(\textbf{X}\geq x_1) = (P(\textbf{X}\leq x_1))^c = 1-P(\textbf{X}\leq x_1)
	\end{equation}
\end{frame}
\section{The NORMDIST function}
	\begin{frame}
		\frametitle{The NORMDIST() function}
		NORMDIST() computes the output of the pdf, $f(x,\mu,\sigma)$,  if \texttt{cumulative} is set to \texttt{FALSE}, or the probability $P(\textbf{X}\leq x)$ if  \texttt{cumulative} is set to \texttt{TRUE}. \\
		The arguments are:
		\begin{itemize}
			\item \texttt{x} : numeric
				\begin{itemize}
					\item The input, $x$, for computing either $f(x,\mu, \sigma)$ or $P(\textbf{X}\leq x)$.
				\end{itemize}
			\item \texttt{mean} : numeric
				\begin{itemize}
					\item $\mu$ the population mean of the normal distribution. 
				\end{itemize}
			\item \texttt{standard\_dev} : numeric
				\begin{itemize}
					\item $\sigma$ the population standard deviation.
					\item Must be 0 or more. 
				\end{itemize}
			\item \texttt{cumulative} : \texttt{TRUE} or \texttt{FALSE}
				\begin{itemize}
					\item \texttt{TRUE} : Computes $P(\textbf{X}\leq x)$.
					\item \texttt{FALSE}: Computes $f(x,\mu, \sigma)$.
				\end{itemize}
		\end{itemize}
		
	\end{frame}

\subsection{Example 1a: Plotting the Normal Distribution}
	\begin{frame}
		\frametitle{Example 1a: Plotting the Normal Distribution}
		Use NORMDIST() to plot heights of men aged 60-69 years, given that their height is on average 5'9" with a standard deviation of 3 inches. Heights are assumed to be normally distributed about the mean. 
		\begin{enumerate}
			\item Label columns A, B, and C as `Height', `Density', and `Cumulative' respectively.  
			\item In cells A2:A20 List the heights increasing in 1in increments from the mean minus 3 standard deviations, (i.e., 60 in) to the mean plus 3 standard deviations (i.e., 78 in).
			\item In cell B2 compute the probability \textit{density} that of male at this height.
			\begin{itemize}
				\item In cell B2 type \texttt{=NORMDIST(A2, 69, 3, FALSE)}
			\end{itemize}
			\item Use Autofill to compute \textit{densities} in cells B3:B20.
		\end{enumerate}
	\end{frame}
	\begin{frame}
		\frametitle{Example 1a: Solution}
		\begin{enumerate}
			\item In cell C2 compute the probability that a male will be this height or shorter. 
			\begin{itemize}
				\item In cell C2 type \texttt{=NORMDIST(A2, 69, 3, TRUE)}
			\end{itemize}
			\item Use Autofill to compute the probabilities in cells C3:C20.
			\item Plot the probability density function and cumulative probabilities using separate 2D-Column charts. 
			\begin{itemize}
				\item Make sure that the column labels reference A2:A20.
				\item Remember to only plot these values as a single series.
			\end{itemize}
		\end{enumerate}
	\end{frame}
	\begin{frame}
		\frametitle{Example 1a: Solution}
		\begin{center}
			\includegraphics[width=0.75\textwidth]{Example1a.png}
		\end{center}
	
	\end{frame}
\subsection{Example 1b: Heights}
	\begin{frame}
		\frametitle{Example 1b: Heights}
		According to US Census Data in 2007/2008 Men aged 60-69 years, are on average 5'9" with a standard deviation of 3 inches. Heights are assumed to be normally distributed about the mean. \\
		\bigskip

		What is the proportion of men who are:
		\begin{enumerate}
			\item Less than and including to 65 inches tall?
			\item Less than and including to 69 inches tall?
			\item Greater than 71 inches tall?
			\item Greater than 68 inches tall?
			\item Between 67 and 70 inches (inclusive)tall? 
		\end{enumerate}
	\end{frame}
	
%	\begin{frame}
%		\frametitle{Notes on NORMDIST()}
%		\begin{itemize}
%			\item Because the Normal Distribution is a continuous distribution, we cannot compute probabilities by counting the individual cases that are in the range asked by the question.
%			\item We must use Cumulative Probabilities (i.e., set the last argument to TRUE).
%				\begin{itemize}
%					\item when set to true the probabilitiy from $-\infty$ up to and including $x$ is computed. 
%				\end{itemize}
%			\item Setting the last argument to \texttt{FALSE} will compute the \textit{output} of the probability density function. $\leftarrow$ This is \textbf{NOT} a Probability.
%		\end{itemize}	
%	\end{frame}
  	\begin{frame}
  		\frametitle{Example 1b: Solution}
  		For Probabilities that include:
  		\begin{itemize}
  			\item \textit{fewer} then compute $P(\textbf{X} \leq x)$.
  				\begin{itemize}
  					\item Example 1b Q1:\\ $P(\textbf{X} \leq 65) $ type: \texttt{=NORMDIST(65, 69, 3, TRUE)}.
  				\end{itemize} 
  			\item \textit{greater} then compute $P(\textbf{X} \geq x)$.  
  			\begin{itemize}
  				\item Use (Eq. 5) to express $P(\textbf{X} \geq x)$ as a complement of $P(\textbf{X}\leq x)$.
  				\item Example 1b Q3: \\$P( \textbf{X} \geq 71) = (P(\textbf{X} \leq x))^c = 1 - P(\textbf{X} \leq x) $ type: \texttt{=1-NORMDIST(71, 69, 3, TRUE)}.
  			\end{itemize}
  		
  			%	\item Use NORMDIST() to compute the probability from $-\infty$ up to the \textit{height} of interest.
  			%	\item Subtract the probability computed from NORMDIST() from 1. 
  			%\end{enumerate}
  			\item \textit{lower}, $x_1$ and  \textit{upper}, $x_2$ bounds then compute $P(x_1 \leq \textbf{X} \leq x_2)$.
  			\begin{itemize}
  				\item Use (Eq. 4) to express the probability as a difference. 
  				\item Example 1b Q5: \\
  				$P(x_1 \leq \textbf{X} \leq x_2) = P(\textbf{X} \leq x_2) - P(\textbf{X} \leq x_1)$ type:\\
  				\footnotesize{ 
  				\texttt{=NORMDIST(70, 69, 3, TRUE) - NORMDIST(67, 69, 3, TRUE)}
  			}
  			\end{itemize}
	\end{itemize}	
\end{frame}

\begin{frame}
	\frametitle{Example 1b: Solution}
	{\footnotesize
		\begin{tabular}{l | l | l| l}
			Range & Expression & Computation & Result \\
			\hline 
			$\left(-\infty,\textbf{65}\right]$  & $P(X \leq \textbf{65})$ & \texttt{=NORMDIST(\textbf{65}, 69, 3, TRUE)} & 0.0912 \\
			$\left(-\infty,\textbf{69}\right]$  & $P(X \leq \textbf{65})$ & \texttt{=NORMDIST(\textbf{69}, 69, 3, TRUE)} & 0.5000 \\
			$\left[\textbf{71}, \infty \right) $ & $P(X\geq\textbf{71})$ & \texttt{=1-NORMDIST(\textbf{71}, 69, 3, TRUE)} &  0.2523\\
			$\left[\textbf{68}, \infty \right) $ & $P(X\geq\textbf{68})$ & \texttt{=1-NORMDIST(\textbf{68}, 69, 3, TRUE)} & 0.6306\\
			$\left[\textbf{67}, \textbf{70}\right]$ &$P(\textbf{67} \leq X \leq \textbf{70})$ & \texttt{=NORMDIST(\textbf{70}, 69,3,TRUE) - }& 0.3781\\
			&&\texttt{  NORMDIST(\textbf{67}, 69,3, TRUE)}&\\
		\end{tabular}
	}
\end{frame}
\section{Facts and Special Cases of the Normal Distribution}
	\begin{frame}
		\frametitle{Facts to Remember about the Normal Distribution}
		\begin{enumerate}
			\item $\sim68\%$ of the area under the curve is enclosed within $\pm1$ standard deviation, $1\sigma$, from the mean, $\mu$.
			\item $\sim95\%$ of the area under the curve is enclosed within $\pm2$ standard deviations, $2\sigma$, from the mean, $\mu$.
			\item $\sim99\%$ of the area under the curve is enclosed within $\pm3$ standard deviations  $3\sigma$,from the mean, $\mu$.
		\end{enumerate}
	\begin{center}
		\includegraphics[width=0.75\textwidth]{areaundercurve.png}
	\end{center}
\end{frame}
	
	\begin{frame}
		\frametitle{The Standard Normal Distribution}
		The Standard Normal Distribution is a special case of the Normal Distribution. where
		\begin{itemize}
			\item The mean, $\mu = 0$
			\item The standard deviation, $\sigma =1$.
		\end{itemize}
		\begin{center}
		\begin{figure}
		\includegraphics[width=0.65\textwidth]{SNorm.png}
		\caption{The Standard Normal Distribution}
		\end{figure}
	\end{center}
	\end{frame}
\section{Further Reading}
	\begin{frame}
		\frametitle{Further Reading}
		The topics covered in the lecture can be found in \textit{Compter Applications for Life Sciences} p. 76-84.
	\end{frame}
	
\end{document}
