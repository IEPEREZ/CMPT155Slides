\documentclass[12pt]{beamer}
\usetheme{Boadilla}
\usepackage{graphicx}
\usepackage{algorithm2e}
\graphicspath{{images/}}
\title{CMPT 155: Computer Applications for Life Sciences}
\subtitle{Lecture 12: Importing, Sorting and Parsing Data; Matrix Operations}
\author{Ivan E. Perez}
\institute{}
\date{April 26, 2022}
\usepackage{booktabs} % Allows the use of \toprule, 
\usepackage{appendix}
\usepackage{enumerate,multicol}
\usepackage{amsmath, amssymb, amsthm}
\usepackage{tikz}
\usepackage{amsxtra}
\usepackage{cancel}
\begin{document}
	
	\begin{frame}
		\titlepage
	\end{frame}
	
	\begin{frame}
		\frametitle{Presentation Outline}
		\tableofcontents
	\end{frame}
	\section{Homework \& Administrative}
	
	\begin{frame}
			\frametitle{Homework \& Administrative Schedule}
		\begin{itemize}
			\item Homeworks:
			\begin{itemize}	
				\item \#7 Due: Friday, April 29\textsuperscript{rd} at 6pm
				\item \#8 Due: Friday, May 6\textsuperscript{th} at 6pm
			\end{itemize}
			\item Final Exam Review: Tuesday, May 3\textsuperscript{rd} at 6pm
			\item Mock Final Exam: Wednesday, May 4\textsuperscript{th}
			\item \textbf{Final Exams}:
			\begin{itemize}
				\item Section 01 (8am) Final Exam: May 9\textsuperscript{th} 11am - 1pm
				\item Section 02 (9am) Final Exam: May 10\textsuperscript{th} 11 am - 1pm
			\end{itemize}
		\end{itemize}
	\end{frame}
\section{Importing, Sorting and Parsing Data}
\subsection{Importing Data}
	\begin{frame}
		\frametitle{Importing Data}
		Data can be imported from:
		\begin{itemize}
			\item text files (e.g., .txt, .csv)  
			\item database connections (e.g., MySQL, MSAccess) 
		\end{itemize}	 
	\end{frame}

	\begin{frame}
		\frametitle{Importing from Text Files}
				Importing Data From Text
		\begin{itemize}
			\item right (Ctrl) -click a text file and try opening it with Excel
			\item In the Data Tab go to (Get Data) followed by From Text
			\item The Text to Columns wizard should start up. 
		\end{itemize}
	\end{frame}

\subsection{Parsing Data}
	\begin{frame}
		\frametitle{Parsing Data}
		Data can be parsed using the Text to Columns wizard. 
		\begin{figure}[ht]
			\begin{minipage}{0.3\linewidth}
				%\centering  % redundant
				\includegraphics[width=\textwidth]{T2CStep1.png}
				\caption{Step 1: Specify the way you wanted to delimit (i.e., separate/find breaks) in your data.}
				\label{fig:figure1}
			\end{minipage}%
			\hfill% not: "\hspace{0.5cm}"
			\begin{minipage}{0.3\linewidth}
				%\centering  % redundant
				\includegraphics[width=\textwidth]{T2CStep2.png}
				\caption{Step 2: Apply the delimiter that makes sense for your raw data. In this case `Commas', `,' is our delimiter.}
				\label{fig:figure2}
			\end{minipage}%
			\hfill% not: "\hspace{0.5cm}"
			\begin{minipage}{0.3\linewidth}
				%\centering  % redundant
				\includegraphics[width=\textwidth]{T2CStep3.png}
				\caption{Step 3: Verify that data has been parsed correctly, and add final touches and/or Advanced options.}
				\label{fig:figure3}
			\end{minipage}
		\end{figure}
	\end{frame}

	\begin{frame}
		\frametitle{Concatenating Data}
		Cells can be concatenaed by using the `\&' operator or using CONCAT().
			\begin{figure}
				\begin{center}
					\includegraphics[width=0.9\textwidth]{CONCAT.png}
				\end{center}
			\end{figure}
	\end{frame}
\subsection{Sorting Data}
	\begin{frame}
		\frametitle{Sorting Data}
		Data can be sorted in the Home $\rightarrow$ Sort \& Filter Menu:
		\begin{itemize}
			\item manually by using `Custom Sort' wizard.
			\item by A-Z by using `Sort A to Z' or `Sort Z to A'.
			\item automatically selecting Filter icon and using the Filter submenus. 
		\end{itemize}
	\end{frame}
	\begin{frame}
		\frametitle{Filtering Data}
		Data can be filtered and sorted using the Autofilter button in the Data Tab.
		
	\end{frame}
\subsection{Example 1: NY Data}
	\begin{frame}
		\frametitle{Example 1: NY data}
		Restated from p.45 
		\begin{enumerate}
			\item Open: \textcolor{blue}{ \href{https://data.cityofnewyork.us/Social-Services/311-Service-Requests-from-2010-to-Present/erm2-nwe9}{NYC Open Data Search 311 Service Requests 2010 to Present}}.
			\item Click Export $\rightarrow$ CSV. A very long download should start.
			\begin{itemize}
			\item If you want to get a feel for the data try using a snippet of this dataset called NYCOpenData311Sample.csv
			\end{itemize}
			\item Try importing this data using the Data import wizard.
			\item Try answering the following questions:
			\begin{enumerate}[a.]%[label=\Alph*]
				\item How Many 311 requests were filed under the Department of Transportation (DOT), and how many were filed under the NYPD?
				\item How many complaints did each Borough(Communitys) have?
				\item What Type of complaint was the most common?
				\item What were the Unique Keys, and Descriptors of the complaints \textbf{not} associated with Noise?
			\end{enumerate}
		\end{enumerate}
	\end{frame}
	\begin{frame}
		\frametitle{Example 1: Solution}
		\begin{center}
			\begin{tabular}{l |l}
				Question No. & Answer \\
				\hline
				a & DOT = 2; NYPD = 12\\
				b & Manhattan = 5; Bronx = 3; \\
				& Brooklyn  =4; Queens =2  \\
				c  & Noise Complaint \\
			\end{tabular}
		\end{center}
	d. 
	\begin{center}
		\begin{tabular}{ l | l }
			Unique Key & Descriptor \\
			\hline
			997177	&	Pothole\\
			53995389 & With License Plate\\
			53994527 & Blocked Hydrant\\
			53999207 & Plate Condition - Noisy\\
			54000934 & Blocked Hydrant
		\end{tabular}
	\end{center}
	\end{frame}
\subsection{Exercise 1: Jordan Sales}
	\begin{frame}
		\frametitle{Exercise 1: Jordan Sales }
		\begin{enumerate}
			\item Import the file `JordanSales.csv'.
			\item Use Autofiler to Create Filter Criteria.
			\item Answer the following Questions about the data set.
			\begin{itemize}
				\item What is the average sneaker sales price for release years 2014 through 2019?
				\item How many options does a customer have if they want a sneaker from 2019 with an average retail price between \$175-\$250?
				\item Based on your taste, what release year would you buy from and how much would you be willing to pay for Jordans?
			\end{itemize}
		\end{enumerate}
	\end{frame}
\section{Matrices}
	\begin{frame}
		\frametitle{Matrices}
		Matrices are arrays of numbers $m$-rows and $n$-columns. 
		Similar to how we performed operations on cells with single values, certain operations and be applied to matrices. 
		Matrices can be labled using capital letters.
		$$ \textbf{\text{A}} = 
			\begin{bmatrix}
				1 & 3 \\
				2 & 6 \\
				7 & 9 
			\end{bmatrix}
			\quad 
			\textbf{\text{B}} = 
			\begin{bmatrix}
				4 & 5 & 8 \\
				10 & 11 & 12 
			\end{bmatrix}
			\quad
			\textbf{\text{C}} = 
			\begin{bmatrix}
				5 & 6 \\
				7 & 8 
			\end{bmatrix}
			\quad
			\textbf{\text{I}} =
			\begin{bmatrix}
				1 & 0 \\
				0 & 1 \\
			\end{bmatrix}
			\quad
		$$
	\end{frame}
\subsection{Matrix Operations}
	\begin{frame}
		\frametitle{Matrix Operations}
			Addition: $+$
			\begin{itemize}
				\item Matrices must be the same size
				\item Size: $(m \times n) - (m\times n)$ 
				\item Output Size: $ m \times n $
				\item Example: $\textbf{\text{A}} + \textbf{\text{A}} = 2\textbf{\text{A}}$ 
			\end{itemize}
			Subtraction: $-$
			\begin{itemize}
				\item Matrices must be the same size
				\item Size: $(m \times n) - (m\times n)$ 
				\item Output Size: $ m \times n $
				\item Example: $\textbf{\text{A}} - \textbf{\text{A}} = 0 $ 
			\end{itemize}
%		\begin{tabular}{| l | l | c | c | c | }
%			Operation & Comment & Size & Size of Result & Example\\
%			\hline
%			Subtraction &Matrices must be the same size & $ & A-A = 0 \\
%			\hline
%			Subtraction &Matrices must be the same size & $(m \times n) + (m\times n)$ & $m\times n$  & \\  
%			\hline
%			Multiplication,& Number of rows in first matrix &  & $m \times r$ & $A\dot B = AB  $\\
%			 `MMULT()' &&&&\\
%			& & & & \\
%			&in the second matrix &&&\\
%			\hline 
%			Inverse, & Square \textit{nonsingular} matrices only!,  & && \\
%			 `MINVERSE()' &&&&\\
%			 \hline
%			Determinant, &&&&\\
%			`MDTEREM()' &&&& \\ 
%			\hline
%			Transpose, &&&&\\
%			`TRANSPOSE()' &&&&\\ 
%			
%		\end{tabular}
%	
	\end{frame}
	\begin{frame}
		\frametitle{Matrix Operations: Continued}
		Multiplication: MMULT()
		\begin{itemize}
			\item  Number of rows in first matrix \textbf{MUST} equal number of columns in the second matrix.
			\item Size: $(m\times n) \cdot (n \times r) $
			\item Output Size: $ m \times r $
			\item Example:  $\textbf{\text{A}} \cdot \textbf{\text{B}} = \textbf{\text{AB}}$ 
		\end{itemize}
		Determinant:  MDETERM() 
		\begin{itemize}
			\item Square matrices only
			\item Size: $n \times n $
			\item Output Size: Single Value
			\item Example: $\text{det}(\textbf{\textbf{C}}) = -2 $
		\end{itemize}
	\end{frame}
	\begin{frame}
		\frametitle{Matrix Operations: Continued}
		 Inverse: MINVERSE()
		 \begin{itemize}
		 	\item Square nonsingular matrices only.
		 	\item Size: $n \times n $
		 	\item Output Size:  $n \times n $
		 	\item Example: $\textbf{\text{C}}^{-1}$
		 \end{itemize}
	 	Transpose: TRANSPOSE()
	 	\begin{itemize}
	 		\item All Matrices
	 		\item Size: $m \times n$ 
	 		\item Output Size: $n \times m$
	 		\item Example: $\textbf{\text{A}}^{\text{T}}$
	 	\end{itemize}
 	\end{frame}
\subsection{Solving Linear Equations}
	\begin{frame}
		\frametitle{Solving Linear Equations with Matrices}
		Linear equations with three unknowns can take the form:
		\begin{align*}
			10x + 12y + 15z = 40 \\
			11x + 12y  + 14z = 80 \\
			3x + 4y + 4z = 24 
		\end{align*}
		Expressing this equation using matrices we get, 
		\begin{equation*}
			\textbf{\text{AX}} = \textbf{\text{b}} 
		\end{equation*}
		Where 
		\begin{equation*}
			\textbf{\text{A}} = 
			\begin{bmatrix}
				10 & 12 & 15 \\
				11 & 12 & 14 \\
				3 & 4 & 4 
			\end{bmatrix}
			\quad
			\textbf{\text{X}} = 
			\begin{bmatrix} 
				x \\
				y \\
				z
			\end{bmatrix}
			\quad
			\textbf{\text{b}} = 
			\begin{bmatrix}
				40 \\
				80 \\
				24
			\end{bmatrix}
		\end{equation*}
	\end{frame}
\subsection{Example 2}
	\begin{frame}
		\frametitle{Example 2: Solution}
		The solution to this equation is 
		\begin{align*}
			\textbf{\text{AXA}}^{-1} =  \textbf{\text{bA}}^{-1}\\
			\textbf{\text{\cancel{A}X}}\cancel{\textbf{\text{A}}^{-1}} =  \textbf{\text{bA}}^{-1}\\
			\textbf{\text{X}} = \textbf{\text{bA}}^{-1}
		\end{align*}
	We can express this solution in Excel by:
	\begin{enumerate}
		\item Writing out arrays for \textbf{A} and \textbf{b}.
		\item Using MINVERSE() on the selection for \textbf{A} to derive \textbf{A}\textsuperscript{-1}.
		\item Using MMULT() to multiply \textbf{b} by \textbf{A}\textsuperscript{-1} .
	\end{enumerate}
	\end{frame}
	\begin{frame}
		\frametitle{Example 2: Solution}
		\begin{center}
			\includegraphics[width=0.9\textwidth]{MatrixExample.png}
		\end{center}
	\end{frame}
\section{Further Reading}
	\begin{frame}
		\frametitle{Further Reading}
		The topics covered in the lecture can be found in \textit{Compter Applications for Life Sciences} p.39-46 and p. 85-90
	\end{frame}

\end{document}
